\documentclass{article}
%List of PACKAGES used
\usepackage[utf8]{inputenc}
\usepackage{graphicx}
\usepackage[dutch]{babel} 
\usepackage{xcolor}
\usepackage{amsmath}
\usepackage{amssymb}
\usepackage{tabularx}
\usepackage{array}
\usepackage{hyperref}
\usepackage{multirow}
\usepackage{csquotes}
\usepackage{float}
\restylefloat{table}
\graphicspath{ {./images/} }


\title{%
  DIY Acoustic Camera \\
  \Minutes group 24 }
\author{Atom Levison, Camiel van der Marel, Sam Moerdijk, Jasper Nierse}
\date{\today}
\begin{document}

\maketitle

\section{Week 1}
04/06/24:
Vandaag hebben we de passen geupdate om het lab in te komen. 
De opstelling is bekeken en werkbaar gemaakt.
Met meneer Sprik besproken hoe we het project aan willen pakken: De eerste dingen om te onderzoeken zijn; Onderzoek doen aan geluid delay per microfoon en hiermee de
calibratie afstellen, de basis van de golfvergelijking doornenem, Fourier (powerspectrum), Beamforming en trillingen in materialen.
Verder is de Github opgezet en werkbaar gemaakt.\\
\\
05/06/24
Vandaag hebben wij een inventarisatie gemaakt om hiemee een planning voor de eerste twee weken te maken.
We hebben onderzocht wat Delay and Sum beamforming is en duidelijk gemaakt wat de begrippen main lobe en side lobe zijn. 
We zijn er achter gekomen, met de hulp van meneer Sprik, dat het oorspronkelijke meetonderzoek (het localiseren van een breuk in een metalen buis) naar alle waarschijnlijkheid niet mogelijk is. 

Na een overleg met meneer Sprik zijn we tot de conclusie gekomen dat het interessant is om onderzoek te doen aan de triangel (waar in de triangel bevindt zich de geluidsbron?).

Hier zijn deze onderzoekspunten uit gekomen:
    * Hoe ziet het golfmodel in een triangel eruit? (Gebogen metalen voorwerp)
    * Hoe meet de AC locatie vanuit een 4x4 grid.
    * Hoe werkt de code;
        - Hoe zien de onder en boventonen van een triangel eruit?
        - Hoe linken we de video aan de audio?\\
\\
06/06/24
Vandaag is er gekeken naar de code van de akoestische camera waar een paar eerste test runs zijn gedaan om te kijken hoe de camera's geluid opnemen. Hierbij moet nog worden gekeken hoe het precies werkt en hoe de data geexporteerd kan worden om vergvolgens te onderzoeken.
Verder is een poging gedaan een wiskundig model op te stellen voor ruimtelijke tijdsverschil bepaling over de gehele 4x4 microfoon opstelling voor een puntbron uitgegaan voor een verweg situatie. 
    Hier zijn we tot de conclusie gekomen dat aan de hand van een ruimtelijke bepaling de stap van één dimensie naar twee niet mogelijk is. Er is geen onderscheid te maken in de richtingen.
Ook is er een eerste testopstelling voor de triangel bedacht om uit te zoeken waar het geluid in de triangel gepproduceert wordt.\\
\\
07/06/24
Vandaag hebben we de wiskunde achter de locatiebepalingen van de geluidsbron (voor nu) afgerond. We realiseerde ons dat het mogelijk zou zijn om de hoeken in twee dimensies
te bepalen door slim gebruik te maken van microfoons die op dezelfde rij, danwel kolom, liggen. Nadat we de wiskunde uitgepluist hadden, zijn we aan de slag gegaan met de
code. We hebben besloten om de code zelf te schrijven, omdat we dan meer controle hebben dan wanneer we domweg kopieren. De code is getest en werkt grotendeels, hoewel er
iets nieuws moet worden verzonnen om het padlengteverschil (tijdsverschil) tussen twee microfoontjes te bepalen. We hadden het idee om simpelweg de afstand tussen de twee
maximale datapunten te pakken, maar dat blijkt niet goed te werken. Mogelijk komt dit ook doordat er te veel ruis in het lab is, dus maandag gaan we proberen te testen op
een stillere locatie. Als dat niet werkt, moeten we een andere manier verzinnen om padlengteverschil te bepalen (zie voor verdere informatie hierover github 
/onzecode/toevoeging.png).

\begin{table}[H]
\begin{tabular}{|p{1.5in}|p{4in}|}
\hline
Date/Time/Place & 07/06/24, 11.00-17.30 \\ \hline
What            & Writing Python code for the mic array \\ \hline
Who             & CM \\ \hline
How             & First a script on paper to determine how it will work. Then writing it out in Python \\ \hline
What's next     & I found that the room I am working has to much background noise. We need to test the code in a quit room. \\ \hline
\end{tabular}
\end{table}

\begin{table}[H]
\begin{tabular}{|p{1.5in}|p{4in}|}
\hline
Date/Time/Place & 07/06/24, 10.00-15.00, Home \\ \hline
What            & Working on a poster draft and listening to the Softmattergroup presentations \\ \hline
Who             & JN \\ \hline
How             & Making a rough sketch on paper from where I made it in Canva  \\ \hline
What's next     & Further expanding on it \\ \hline
\end{tabular}
\end{table}

\section{10/06/24}

\begin{table}[H]
\begin{tabular}{|p{1.5in}|p{4in}|}
\hline
Date/Time/Place &  10/06/24, 11.00-11.15/SP\\ \hline
What            &  Consultation sir Sprik\\ \hline
Who             &  AL, CM, SM, JN\\ \hline
How             &  A quick consultation what we will be doing on 10/06.  \\ \hline
What's next     &  Testing of code\\ \hline
\end{tabular}
\end{table}


\begin{table}[H]
\begin{tabular}{|p{1.5in}|p{4in}|}
\hline
Date/Time/Place & 10/06/24, 11.30-12.00/SP \\ \hline
What            &  Testing of code\\ \hline
Who             &  AL, CM, SM, JN\\ \hline
How             &  By testing the code for the acoustic camera in a room with little background noise we tried to determine if the code would work. We found that during the first few test runs the camera picked up a lot of background noise and by reducing the noise we found that the code written by CM functioned. The test setup was determined on location and no data obtained during testing will be used further. \\ \hline
What's next     &  The code for the mics doesn't work as it should and more coding by CM and AL is needed\\ \hline
\end{tabular}
\end{table}
\begin{figure}[H]
    \centering
    \includegraphics[width=8cm, angle =270]{20240610_113743.jpg}
    \caption{Setup during testing}   
\end{figure}

%Hier staat de foto van de opstelling.

\begin{table}[H]
\begin{tabular}{|p{1.5in}|p{4in}|}
\hline
Date/Time/Place &  10/06/24, 12.00-13.30/SP\\ \hline
What            &  Making a Minutes layout\\ \hline
Who             &  JN\\ \hline
How             &  Creating a structure for the minutes in Latex and linking this to GitHub\\ \hline
What's next     &  Making sure that everyone fills in this file so we have a complete picture of what we have done together with the GitHub Planning To Do list.\\ \hline
\end{tabular}
\end{table}

\begin{table}[H]
\begin{tabular}{|p{1.5in}|p{4in}|}
\hline
Date/Time/Place & 10/06/24, 16.30-17.30, At home \\ \hline
What            &  Creating a design for the poster\\ \hline
Who             &  JN \\ \hline
How             &  Completing the design for the poster in Canva\\ \hline
What's next     &  Filling in the poster with text and images after we finish the experiment.\\ \hline
\end{tabular}
\end{table}

\begin{figure}[H]
    \centering
    \includegraphics[width=6cm]{PosterV1.png}
    \caption{Version 1 of the empty poster with filler image.}   
\end{figure}
%Hier staat de foto van de eerste versie van de lege poster.

\begin{table}[H]
\begin{tabular}{|p{1.5in}|p{4in}|}
\hline
Date/Time/Place &  10/06/24, 12.00-13.30, SP\\ \hline
What            &  Studying literature on the triangle \\ \hline
Who             &  SM\\ \hline
How             &  With literature (Github: literature, Triangle.zip) added to Github by Sprik, I tried to educate myself more about the triangle and its sound behavior.\\ \hline
What's next     &   Error propagation. And eventually, collect data from testing and analyze them to understand noise behavior.\\ \hline
\end{tabular}
\end{table}

\begin{table}[H]
\begin{tabular}{|p{1.5in}|p{4in}|}
\hline
Date/Time/Place &  10/06/24, 13.30-14.00, SP\\ \hline
What            &  Error propagation\\ \hline
Who             &  SM\\ \hline
How             &   To calculate the error propagation, I use the partial derivatives of theta to the variables. Then I multiply the partial derivative of theta to a variable by the change in that variable and square it: $$(\frac{\partial\theta}{\partial \text{variable}}\cdot\Delta\text{variable})^2$$ I do this for all variables and then finally take the root of the sum of these squared terms.\\ \hline
What's next     &  \\ \hline
\end{tabular}
\end{table}

\begin{table}[H]
\begin{tabular}{|p{1.5in}|p{4in}|}
\hline
Date/Time/Place &  10/06/24, 11:00 - 14:00, SP\\ \hline
What            &  Fixed a bunch of code\\ \hline
Who             &  AL, CM\\ \hline
How             &  while crying.\\ \hline
What's next     &  more crying\\ \hline
\end{tabular}
\end{table}

\section{11/06/24}

\begin{table}[H]
\begin{tabular}{|p{1.5in}|p{4in}|}
\hline
Date/Time/Place &  11/06/24, 11.00-13.30, SP\\ \hline
What            &  Researching the specs of the camera used and describing how to find the soundsource\\ \hline
Who             &  JN\\ \hline
How             &  By looking up the specs on the Raspberry Pi website and making small sketches of the situation it was possible to find a way to implement the found angles with the microphones code and a picture taken with the camera on the array. \\ \hline
What's next     &  Writing some code to actually implement the camera to the existing code written by AL and CM.\\ \hline
\end{tabular}
\end{table}

\begin{figure}[H]
    \centering
    \includegraphics[width=7cm]{Foto tekening sensor en F.O.V..jpg}
    \caption{Sketch of the f.o.v. of the camera: Raspberry Pi Camera Module 3 Wide}   
\end{figure}
%Hier staat de foto van schets van de camera f.o.v.


\begin{table}[H]
\begin{tabular}{|p{1.5in}|p{4in}|}
\hline
Date/Time/Place & 11/06/24, 11.00-14.00 / 15.30-16.30, SP \\ \hline
What            & Implementing error propagation and adding filters to the code for the array. \\ \hline
Who             & CM, AL, SM \\ \hline
How             &  \\ \hline
What's next     & Finalising the code, linking the camera to the array and testing \\ \hline
\end{tabular}
\end{table}

\section{12/06/24}

\begin{table}[H]
\begin{tabular}{|p{1.5in}|p{4in}|}
\hline
Date/Time/Place & 12/06/24, 9.00-9.30, SP \\ \hline
What            & meeting planning what to do today \\ \hline
Who             & CM, SM, JN \\ \hline
How             & A quick meeting to determine what needs to be done today and dividing who does what \\ \hline
What's next     &  getting to work\\ \hline
\end{tabular}
\end{table}

\begin{table}[H]
\begin{tabular}{|p{1.5in}|p{4in}|}
\hline
Date/Time/Place & 12/06/24, 9.30-13.00, SP \\ \hline
What            & Calculations and code to convert calculated angles to pixel coordinates. \\ \hline
Who             & AL, CM, SM, JN \\ \hline
How             & JN first made a calculation on paper with the found camera specs. First a ratio pixels to the field of view. These ratio's were converted to a classical coordinate grid from where it was transformed to the coordinates used in the image produced by Python. Then these calculations were turned into code by AL, SM and CM, moreover a code was made with the PIL package to show the spot in the image where the detected sound would be coming from.\\ \hline
What's next     & Trying it out. \\ \hline
\end{tabular}
\end{table}

\begin{table}[H]
\begin{tabular}{|p{1.5in}|p{4in}|}
\hline
Date/Time/Place &  12/06/24, 14.00-15.00, SP\\ \hline
What            & Trying the code in a room with less noise \\ \hline
Who             &  AL, CM, SM, JN\\ \hline
How             & By following the "onderzoeksplan" we will test the code and the hardware. We found that the code doesn't work. \\ \hline
What's next     & Refining the code and making it work. \\ \hline
\end{tabular}
\end{table}

\begin{figure}[H]
    \centering
    \includegraphics[width=7cm]{savedimage.jpg}
    \caption{Test photo taken by the array.}   
\end{figure}
%Hier staat de foto van een test genomen door de AC.

\begin{table}[H]
\begin{tabular}{|p{1.5in}|p{4in}|}
\hline
Date/Time/Place & 12/06/24, 15.00-16.00, SP \\ \hline
What            &  Fixing the code\\ \hline
Who             &  AL, CM, SM, JN\\ \hline
How             & By trial and error we tried fixing the code. We found that Python doesn't like the resolution from the image made by the "Raspberry Pi Camera Module 3 Wide" so we did the calculation for the scaling again and implemented this in Python. Furthermore the angle given by Python doesn't make sense in the context we did the experiment in.   \\ \hline
What's next     & Fixing the code \\ \hline
\end{tabular}
\end{table}

\begin{table}[H]
\begin{tabular}{|p{1.5in}|p{4in}|}
\hline
Date/Time/Place & 12/06/24, 17.00-19.30, SP \\ \hline
What            & Trying to fix the code \\ \hline
Who             & CM \\ \hline
How             & Recalculating the pixel ratio to the field of view. After that small bug fixes. \\ \hline
What's next     & Further fixing by trial and error.  \\ \hline
\end{tabular}
\end{table}

\section{13/06/24}

\begin{table}[H]
\begin{tabular}{|p{1.5in}|p{4in}|}
\hline
Date/Time/Place &  13/06/24, 09.00-12.00\\ \hline
What            &  testing animation software\\ \hline
Who             &  SM, AL\\ \hline
How             &  We just looked around for some good software. We also did some research about the best way to make educational videos\\ \hline
What's next     & Making a story board\\ \hline
\end{tabular}
\end{table}

\begin{table}[H]
\begin{tabular}{|p{1.5in}|p{4in}|}
\hline
Date/Time/Place &  13/06/24, 09.00-10.00\\ \hline
What            &  Looking and trying the code\\ \hline
Who             &  SM, AL\\ \hline
How             &  We tried to make some fixes but it didn't work.\\ \hline
What's next     &  praying\\ \hline
\end{tabular}
\end{table}

\begin{table}[H]
\begin{tabular}{|p{1.5in}|p{4in}|}
\hline
Date/Time/Place & 13/06/24, 11.45-12.10, SP \\ \hline
What            & Meeting with mister Sprik \\ \hline
Who             & AL, SM \\ \hline
How             & Mister Sprik gave advice on the code and told us that by trying to use a steady frequency produced (for instance by a phone) and trying to localize this. The triangle produces a specific frequency for a to short amount of time for the code written at the moment. The frequency produced each time hitting the triangle is also not precisely the same which results in faulty measurements.   \\ \hline
What's next     & Making code that localises peaks. By shifting these peaks to make them align it would be possible to find the $$\Delta t.$$ \\ \hline
\end{tabular}
\end{table}

\begin{table}[H]
\begin{tabular}{|p{1.5in}|p{4in}|}
\hline
Date/Time/Place & 13/06/24, 12.00-15.00, SP \\ \hline
What            & Trying to implement the advice given by mister Sprik and trying to fix the code \\ \hline
Who             & SM, CM, JN \\ \hline
How             & By trial and error trying to find where the error may lie. We did this by playing a constant sound of 1900HZ through a phone on different random locations around the array. \\ \hline
What's next     &  The results for the output angles for the plot of the image didn't line up with the location of the source. Also the spot marked as the source kept being placed in the centre of the image. CM's reasoning for this was that the code written for the sound peak filtering/detection is faulty  \\ \hline
\end{tabular}
\end{table}

\begin{table}[H]
\begin{tabular}{|p{1.5in}|p{4in}|}
\hline
Date/Time/Place & 13/06/24, 15.00-16.00, SP \\ \hline
What            &  Programming\\ \hline
Who             &  CM\\ \hline
How             & Finding the average over a bunch of mics instead of two. \\ \hline
What's next     &  Expanding on this and crying \\ \hline
\end{tabular}
\end{table}

\begin{table}[H]
\begin{tabular}{|p{1.5in}|p{4in}|}
\hline
Date/Time/Place & 13/06/24, 15.00-16.00, SP \\ \hline
What            & Studying literature wave equation in matter \\ \hline
Who             & SM \\ \hline
How             & Studying Foerier and transverse waves to implement this knowledge in the poster \\ \hline
What's next     & more studying. \\ \hline
\end{tabular}
\end{table}

\begin{table}[H]
\begin{tabular}{|p{1.5in}|p{4in}|}
\hline
Date/Time/Place & 13/06/24, 15.00-16.00, SP \\ \hline
What            & Writing introduction for poster \\ \hline
Who             & JN \\ \hline
How             & Making a writing plan for the poster. \\ \hline
What's next     & Actually writing it and putting it in the poster \\ \hline
\end{tabular}
\end{table}


\section{14/06/24}


\begin{table}[H]
\begin{tabular}{|p{1.5in}|p{4in}|}
\hline
Date/Time/Place & 14/06/24, 9.30-10.30/11.00-12.00, SP \\ \hline
What            & Writing Introduction for poster \\ \hline
Who             & JN \\ \hline
How             & Using the structure for the introduction JN will write the introduction text for the poster consisting of: Introduction, short explanation of the AC, Our research question and our hypothesis \\ \hline
What's next     & Writing the method using mostly small images. \\ \hline
\end{tabular}
\end{table}

\begin{table}[H]
\begin{tabular}{|p{1.5in}|p{4in}|}
\hline
Date/Time/Place & 14/06/24, 10.00-15.00, SP \\ \hline
What            &  Working on the animation and story board\\ \hline
Who             & AL \\ \hline
How             & With drawing software krita  \\ \hline
What's next     &  animating\\ \hline
\end{tabular}
\end{table}

\section{15/06/24}

\begin{table}[H]
\begin{tabular}{|p{1.5in}|p{4in}|}
\hline
Date/Time/Place & 15/06/24, 11.00-12.45/14.00-15.00, Thuis \\ \hline
What            & Working on poster \\ \hline
Who             & JN \\ \hline
How             & Fixing text and working on method \\ \hline
What's next     & Continuing working on method \\ \hline
\end{tabular}
\end{table}

\section{17/06/24}

\begin{table}[H]
\begin{tabular}{|p{1.5in}|p{4in}|}
\hline
Date/Time/Place & 17/06/24, 9.00-12.45/18.00-19.15, SP/Home \\ \hline
What            & Finalising introduction and writing methods \\ \hline
Who             & JN \\ \hline
How             & Using the sketch on paper as reference to implement it into the poster \\ \hline
What's next     & Finalising methods and working on results and discussion. \\ \hline
\end{tabular}
\end{table}

\begin{figure}[H]
    \centering
    \includegraphics[width=7cm]{inprogressposterV2.png}
    \caption{In progress poster}   
\end{figure}
%Hier staat de foto van een progressie afbeelding poster

\begin{table}[H]
\begin{tabular}{|p{1.5in}|p{4in}|}
\hline
Date/Time/Place & 17/06/24, 11.00-17.00, SP \\ \hline
What            & Fixing code, writing about the way a solid metal tube vibrates and working on script for the animation \\ \hline
Who             & CM,\\ \hline
How             & By now calculating the average delta time over the whole array the amount of math error's has been reduced. \\ \hline
What's next     & Implementing theory in the poster, running final tests on the code and collecting the data.  \\ \hline
\end{tabular}
\end{table}

\begin{table}[H]
\begin{tabular}{|p{1.5in}|p{4in}|}
\hline
Date/Time/Place & 17/06/24, 12.30-17.00, SP \\ \hline
What            & Working on animation \\ \hline
Who             & SM, AL, CM \\ \hline
How             & Making drafts for the animation and writing the script \\ \hline
What's next     & Further work on the animation \\ \hline
\end{tabular}
\end{table}

\begin{figure}[H]
    \centering
    \includegraphics[width=7cm]{Sketch animatie.jpg}
    \caption{sketches for the animation}   
\end{figure}
%Hier staan een paar schetsen voor in de animatie

\begin{table}[H]
\begin{tabular}{|p{1.5in}|p{4in}|}
\hline
Date/Time/Place & 17/06/24, 16.15-17.15, Home \\ \hline
What            & Working on the minutes \\ \hline
Who             &  JN\\ \hline
How             & Writing the minutes for 17/06/24 \\ \hline
What's next     &  \\ \hline
\end{tabular}
\end{table}

\section{18/06/24}

\begin{table}[H]
\begin{tabular}{|p{1.5in}|p{4in}|}
\hline
Date/Time/Place & 18/06/24, 9.00-11.00, SP \\ \hline
What            & Collecting data and testing code/fixing code \\ \hline
Who             & AL, CM, SM, Jn \\ \hline
How             & Finding a room with little noise pollution and doing test runs\\ \hline
What's next     & It doesn't work anymore, quick council with sir Sprik and praying that it will work again later today.\\ \hline
\end{tabular}
\end{table}

\begin{table}[H]
\begin{tabular}{|p{1.5in}|p{4in}|}
\hline
Date/Time/Place & 18/06/24, 11.00-11.45, SP \\ \hline
What            & Meeting with sir Sprik to show what we have and still have to do \\ \hline
Who             & AL, CM, SM, Jn\\ \hline
How             & Asking for feedback on our work progress\\ \hline
What's next     & Implementing this feedback\\ \hline
\end{tabular}
\end{table}

\begin{table}[H]
\begin{tabular}{|p{1.5in}|p{4in}|}
\hline
Date/Time/Place & 18/06/24, 12.00-13.30/18.45-19.45, SP/Home \\ \hline
What            & Working on poster  \\ \hline
Who             & JN \\ \hline
How             & Finishing method \\ \hline
What's next     & Adding final picture in method and working on discussion. \\ \hline
\end{tabular}
\end{table}

\begin{table}[H]
\begin{tabular}{|p{1.5in}|p{4in}|}
\hline
Date/Time/Place & 18/06/24, 11.30-16.30, SP\\ \hline
What            &  working on the animation\\ \hline
Who             &  SM, AL\\ \hline
How             &  drawing and animating\\ \hline
What's next     &  the same\\ \hline
\end{tabular}
\end{table}

\begin{table}[H]
\begin{tabular}{|p{1.5in}|p{4in}|}
\hline
Date/Time/Place &  18/06/24, 11:30-15:00, SP\\ \hline
What            &  Trying to understand and write theory section for poster\\ \hline
Who             &  CM\\ \hline
How             &  Reading some articles, worked some stuff out myself, discussed with Rudolf a little\\ \hline
What's next     &  Writing another section of poster\\ \hline
\end{tabular}
\end{table}

\begin{table}[H]
\begin{tabular}{|p{1.5in}|p{4in}|}
\hline
Date/Time/Place &  18/06/24, 15:00-17:00, SP\\ \hline
What            &  Fixing code of this morning\\ \hline
Who             &  CM\\ \hline
How             &  Bashing head against table\\ \hline
What's next     &  Try to record again tomorrow\\ \hline
\end{tabular}
\end{table}

\section{19/06/24}

\begin{table}[H]
\begin{tabular}{|p{1.5in}|p{4in}|}
\hline
Date/Time/Place & 19/06/24, 8:00-10:00, Sp \\ \hline
What            & Discussing script, working on animation and poster \\ \hline
Who             & CM, SM, JN, AL \\ \hline
How             & Combining our brainpower and collective creativity to work on the script for the animation, the animation and the poster. \\ \hline
What's next     & Just continuing as we are and hopefully finally getting good data  \\ \hline
\end{tabular}
\end{table}

\begin{table}[H]
\begin{tabular}{|p{1.5in}|p{4in}|}
\hline
Date/Time/Place & 19/06/24, 10:00-11:00, Sp \\ \hline
What            & Doing final tests with code and collecting data \\ \hline
Who             & CM, SM, JN, AL \\ \hline
How             & Doing tests in a room with little noise pollution.  \\ \hline
What's next     & Using the found data and implement it into the poster \\ \hline
\end{tabular}
\end{table}

\begin{figure}[H]
    \centering
    \includegraphics[width=7cm]{finalpicture.png}
    \caption{Successful test results}   
\end{figure}
%Hier staat een foto van een succesvol data verzameling

\begin{table}[H]
\begin{tabular}{|p{1.5in}|p{4in}|}
\hline
Date/Time/Place & 19/06/24, 11:00-15:30, Sp \\ \hline
What            & Working on poster \\ \hline
Who             & CM, JN \\ \hline
How             & Writing and fixing text for poster \\ \hline
What's next     & Continuing work on poster \\ \hline
\end{tabular}
\end{table}

\begin{table}[H]
\begin{tabular}{|p{1.5in}|p{4in}|}
\hline
Date/Time/Place &  \\ \hline
What            &  \\ \hline
Who             &  \\ \hline
How             &  \\ \hline
What's next     &  \\ \hline
\end{tabular}
\end{table}

\begin{table}[H]
\begin{tabular}{|p{1.5in}|p{4in}|}
\hline
Date/Time/Place &  \\ \hline
What            &  \\ \hline
Who             &  \\ \hline
How             &  \\ \hline
What's next     &  \\ \hline
\end{tabular}
\end{table}

\begin{table}[H]
\begin{tabular}{|p{1.5in}|p{4in}|}
\hline
Date/Time/Place &  \\ \hline
What            &  \\ \hline
Who             &  \\ \hline
How             &  \\ \hline
What's next     &  \\ \hline
\end{tabular}
\end{table}

\begin{table}[H]
\begin{tabular}{|p{1.5in}|p{4in}|}
\hline
Date/Time/Place &  \\ \hline
What            &  \\ \hline
Who             &  \\ \hline
How             &  \\ \hline
What's next     &  \\ \hline
\end{tabular}
\end{table}

\begin{table}[H]
\begin{tabular}{|p{1.5in}|p{4in}|}
\hline
Date/Time/Place &  \\ \hline
What            &  \\ \hline
Who             &  \\ \hline
How             &  \\ \hline
What's next     &  \\ \hline
\end{tabular}
\end{table}

\begin{table}[H]
\begin{tabular}{|p{1.5in}|p{4in}|}
\hline
Date/Time/Place &  \\ \hline
What            &  \\ \hline
Who             &  \\ \hline
How             &  \\ \hline
What's next     &  \\ \hline
\end{tabular}
\end{table}

\begin{table}[H]
\begin{tabular}{|p{1.5in}|p{4in}|}
\hline
Date/Time/Place &  \\ \hline
What            &  \\ \hline
Who             &  \\ \hline
How             &  \\ \hline
What's next     &  \\ \hline
\end{tabular}
\end{table}

\begin{table}[H]
\begin{tabular}{|p{1.5in}|p{4in}|}
\hline
Date/Time/Place &  \\ \hline
What            &  \\ \hline
Who             &  \\ \hline
How             &  \\ \hline
What's next     &  \\ \hline
\end{tabular}
\end{table}

\begin{table}[H]
\begin{tabular}{|p{1.5in}|p{4in}|}
\hline
Date/Time/Place &  \\ \hline
What            &  \\ \hline
Who             &  \\ \hline
How             &  \\ \hline
What's next     &  \\ \hline
\end{tabular}
\end{table}

\begin{table}[H]
\begin{tabular}{|p{1.5in}|p{4in}|}
\hline
Date/Time/Place &  \\ \hline
What            &  \\ \hline
Who             &  \\ \hline
How             &  \\ \hline
What's next     &  \\ \hline
\end{tabular}
\end{table}

\begin{table}[H]
\begin{tabular}{|p{1.5in}|p{4in}|}
\hline
Date/Time/Place &  \\ \hline
What            &  \\ \hline
Who             &  \\ \hline
How             &  \\ \hline
What's next     &  \\ \hline
\end{tabular}
\end{table}

\begin{table}[H]
\begin{tabular}{|p{1.5in}|p{4in}|}
\hline
Date/Time/Place &  \\ \hline
What            &  \\ \hline
Who             &  \\ \hline
How             &  \\ \hline
What's next     &  \\ \hline
\end{tabular}
\end{table}

\begin{table}[H]
\begin{tabular}{|p{1.5in}|p{4in}|}
\hline
Date/Time/Place &  \\ \hline
What            &  \\ \hline
Who             &  \\ \hline
How             &  \\ \hline
What's next     &  \\ \hline
\end{tabular}
\end{table}

\end{document}


